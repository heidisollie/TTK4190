\section{Linearization of the satellite model}

The equations of motion for our satellite modell are as follows:
\begin{equation*}
\label{eq:dynamics}
	\begin{aligned}
		\dot{\mathbf{q}} = \mathbf{T}_q (\mathbf{q} ) \boldsymbol{\omega} \\
		\mathbf{I}_{CG} \dot{\boldsymbol{\omega}} - \mathbf{S} (\mathbf{I}_{CG} \boldsymbol{\omega} ) \boldsymbol{\omega} & =  \boldsymbol{\tau}
	\end{aligned}	
\end{equation*}
with inertia matrix $\mathbf{I}_{CG} = mr^2 \mathbf{I}_3$, $m = 100$ kg, $r = 2.0$ m.

Let $\mathbf{x = \m{\boldsymbol{\epsilon} \\ \boldsymbol{\omega}}}$. We want to find the equilibrium point $\mathbf{x_0}$ corresponding to $$\mathbf{q^*} = \m{\eta^* \\ \epsilon_1^* \\ \epsilon_2^* \\ \epsilon_3^*} = \m{1 \\ 0 \\ 0 \\ 0} \text{and } \boldsymbol{\tau = \mathbf{0}}.$$ \\
Let's take a look at the first equation of motion
\begin{equation*}
    \mathbf{\dot{q} = T_q(q)\boldsymbol{\omega}}
\end{equation*} 

From equation (2.68) in Fossen\cite{bok} we know this can be expanded to
\begin{equation*}
    \mathbf{\dot q} = \frac{1}{2} \m{-\boldsymbol{\epsilon^\top} \\ \eta \mathbf{I}_3 + \mathbf{S}(\boldsymbol{\epsilon})}\boldsymbol{\omega}
\end{equation*} 

We know the equilibrium point, $\mathbf{x_0 = x^*}$, is a point that satisfies the following equation:
\begin{equation*}
    \mathbf{\dot q = T_q(q^*) \\ \boldsymbol{\omega^*} = 0 }
\end{equation*}


\begin{equation*}
    \begin{split}
        &\mathbf{T_q(q^*)}\boldsymbol{\omega^*} = \mathbf{0} \\
        &\Rightarrow \frac{1}{2}
        \m{
            -\boldsymbol{\epsilon^\top}\\
            \eta \mathbf{I_{3}} + \mathb{S}(\boldsymbol{\epsilon})
        }
        \boldsymbol{\omega^*} = \mathbf{0}\\
        &\Rightarrow \frac{1}{2}
        \m{
            -\epsilon_1^*   &  -\epsilon_2^*   &   -\epsilon_3^*\\
            \eta^*   &   -\epsilon_3^*   &   \epsilon_2^*\\
            \epsilon_3^*   &   \eta^*   &   -\epsilon_1^*\\
            -\epsilon_2^*   &   \epsilon_1^*   &   \eta^*\\  
        }
        \boldsymbol{\omega^*} = \mathbf{0}\\
        &\Rightarrow \frac{1}{2}
        \m{
            0   &   0   &   0\\
            1   &   0   &   0\\
            0   &   1   &   0\\
            0   &   0   &   1\\  
        }
        \boldsymbol{\omega^*} = \mathbf{0}\\
        &\Rightarrow \boldsymbol{\omega^*} = \m{\omega_1^* \\ \omega_2^* \\ \omega_3^*} = \m{0 \\ 0 \\ 0 }
    \end{split}
\end{equation*}

The equilibrium point $\mathbf{x_0 = x^*} = [\epsilon_1^*,\epsilon_2^*,\epsilon_3^*,\omega_1^*,\omega_2^*,\omega_3^*]^\top = [0,0,0,0,0,0]^\top$ \\ \\ \\
We now want to linearize the model about the equilibrium point. That is, we want to define the new state $\Delta \mathbf {x}$ and the new control input $\Delta \boldsymbol{\tau}$ and put the model on the following form:
\begin{equation*}
    \Delta \mathbf{\dot x} = \mathbf{A} \Delta\mathbf{x} + \mathbf{B} \Delta \boldsymbol{ \tau}
\end{equation*}
where $\mathbf{A}$ and $\mathbf{B}$ are defined below.

\begin{equation*}
    \mathbf{A} =
    \begin{bmatrix}
        \frac{\delta\dot{x}_1}{\delta x_1}  &   \cdots  &   \frac{\delta\dot{x}_1}{\delta x_6}\\
        \vdots                              &   \ddots  &   \vdots\\
        \frac{\delta\dot{x}_6}{\delta x_1}  &   \cdots  &   \frac{\delta\dot{x}_6}{\delta x_6}
    \end{bmatrix}\Biggr|_{\mathbf{x=x_0}}
    \text{and  }
    \mathbf{B} =
    \begin{bmatrix}
        \frac{\delta\dot{x}_1}{\delta \tau_1}  &   \cdots  &   \frac{\delta\dot{x}_1}{\delta\tau_3}\\
        \vdots                              &   \ddots  &   \vdots\\
        \frac{\delta\dot{x}_6}{\delta \tau_1}  &   \cdots  &   \frac{\delta\dot{x}_6}{\delta \tau_3}
    \end{bmatrix}\Biggr|_{\mathbf{x=x_0}}
\end{equation*} \\ \\
Now let's take a look at the second equation of motion
\begin{equation*}
    \mathbf{I_{CG}}\dot{\boldsymbol{\omega}} - \mathbf{S(I_{CG}}\boldsymbol{\omega})\boldsymbol{\omega} = \boldsymbol{\tau}
\end{equation*}

The cross product of a vector with itself, $\boldsymbol{\omega}\times\boldsymbol{\omega} = \mathbf{S}(\boldsymbol{\omega})\boldsymbol{\omega}$, is always equal to zero. We can use this to simplify the second equation of motion:

\begin{equation*}
    \mathbf{I_{CG}} \dot{\boldsymbol{\omega}} = \boldsymbol{\tau}\\
    \Rightarrow \dot{\boldsymbol{\omega}} = \mathbf{I_{CG}}^{-1}\boldsymbol{\tau} = \m{\frac{1}{mr^2} & 0 & 0 \\ 0 & \frac{1}{mr^2} & 0 \\ 0 & 0 & \frac{1}{mr^2}} \boldsymbol{\tau}
\end{equation*}
Then we can use the following result  to linearize the model:
\begin{equation*}
\m{\dot x_1 \\ \dot x_2 \\ \dot x_3 \\ \dot x_4 \\ \dot x_5 \\ \dot x_6 } 
=\m{\dot \epsilon_1 \\ \dot \epsilon_2 \\ \dot \epsilon_3 \\ \dot \omega_1 \\ \dot \omega_2 \\ \dot \omega_3} 
= \m {\frac{1}{2}(\omega_1 \eta - \omega_2 \epsilon_3 + \omega_3 \epsilon_2) \\ \frac{1}{2} (\omega_1 \epsilon_3 + \eta \omega_2 - \epsilon_1 \omega_3) \\ \frac{1}{2}(-\omega_1 \epsilon_2 + \omega_2 \epsilon_1 + \eta \omega_3) \\ \frac{1}{mr^2}\tau_1 \\ \frac{1}{mr^2}\tau_2 \\ \frac{1}{mr^2}\tau_3}
\end{equation*}


\begin{equation*}
    \mathbf{A} =  \m{
        0   &   0   &   0   &   \frac{1}{2} &   0           &   0\\
        0   &   0   &   0   &   0           &   \frac{1}{2} &   0\\
        0   &   0   &   0   &   0           &   0           &   \frac{1}{2}\\
        0   &   0   &   0   &   0           &   0           &   0\\
        0   &   0   &   0   &   0           &   0           &   0\\
        0   &   0   &   0   &   0           &   0           &   0\\}
    \text{and  }
    \mathbf{B} = \m{
        0       &   0       &   0       \\
        0       &   0       &   0       \\
        0       &   0       &   0       \\
        \frac{1}{mr^2}  &   0       &   0       \\
        0       &   \frac{1}{mr^2}  &   0       \\
        0       &   0       &   \frac{1}{mr^2}  \\}
\end{equation*}

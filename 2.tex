\section{Introducing the first controller}
 
Now we will consider the attitude control law
\begin{equation*}
    \boldsymbol{\tau} = -\mathbf{K}_d \boldsymbol{\omega} - k_p \boldsymbol{\epsilon}, \mathbf{K_d} = k_d \mathbf{I}_3, k_p > 0, k_d > 0
\end{equation*}
Let $k_d = 20$, $k_p = 1$. \\
To make things easier, intuitively, we will write the law on the form $\boldsymbol{\tau} = \mathbf{-Kx}$
\begin{equations}
    \begin{align*}
        -\mathbf{Kx} &= -\mathbf{K_d} \boldsymbol{\omega} -k_p\boldsymbol{\epsilon}\\
        &=
        \begin{bmatrix}
            -k_d\omega_1 -k_p\epsilon_1\\
            -k_d\omega_2 -k_p\epsilon_2\\
            -k_d\omega_3 -k_p\epsilon_3
        \end{bmatrix}\\
        \Rightarrow \mathbf{K} &=
        \begin{bmatrix}
            k_d  &   0   &   0   &   k_p   &   0   &   0\\
            0   &   k_d  &   0   &   0   &   k_p   &   0\\
            0   &   0   &   k_d  &   0   &   0   &   k_p\\
        \end{bmatrix}
    \end{align*}
\end{equations} \\

If we insert this result into our linearized model we get
\begin{equation*}
    \mathbf{\dot x} = \mathbf{Ax+B}\boldsymbol{\tau} = \mathbf{Ax + B(-Kx)} = \mathbf{(A-BK)x}
\end{equation*}
Note: to make things easier to write we will be describing the states and control input of our linearized model as $\mathbf{x}$ and $\boldsymbol{\tau}$. \\

To verify that the closed-loop system is stable, we need to show that the eigenvalues of the system matrix ($\mathbf{A-BK}$) lie in the left half-plane of the complex plane:

\begin{equation*}
    \mathbf{A-BK} =
    \begin{bmatrix}
        0           &       0           &   0           &       1/2     &       0       &        0       \\
        0           &       0           &   0           &       0       &       1/2     &        0       \\
        0           &       0           &   0           &       0       &       0       &       1/2     \\
        -\frac{20}{mr^2}&   0           &   0           &   -\frac{1}{mr^2}&    0       &        0       \\
        0           &   -\frac{20}{mr^2}&   0           &       0       &   -\frac{1}{mr^2}&     0       \\
        0           &       0           &   -\frac{20}{mr^2}&   0       &       0       &   -\frac{1}{mr^2} \\
    \end{bmatrix}
\end{equation*}

With the help of MATLAB, we find the eigenvalues of $\mathbf{A-BK}$:

\begin{equation*}
    \lambda_{1,2,3,4,5,6} = -0.025 \pm 0.025i
\end{equation*}

The eigenvalues lie in the left half-plane, so the closed-loop system with $k_p = 1$ and $k_d = 20$ is stable.

A control system with complex poles reaches the desired state faster than a system with real poles, but the imaginary part of complex poles adds oscillations to the system response. In space, we wish to have a system as stable as possible, with minimal oscillations, due to no friction. \\
On the other hand, real parts requires more power to give the same response speed as complex poles. Therefore, we would prefer to have complex poles $\alpha \pm j\beta$ with small $\beta$, and hence minimal oscillations. 